\subsubsection{Порядок выбора пакета при~разрешении \EN{requires}}

При~установке пакетов в~рамках обработки зависимостей требуется  просмотр известного множества записей \EN{obsoletes}.
При~этом требуемый пакет может быть заменён на~пакет с~соответствующей записью \EN{obsoletes} по~следующим правилам:

\begin{enumerate}

\item {
Замена производится только в~случае, если пакет не~установлен или ограничение версии в~записи \EN{requires} не~соответствует версии установленного пакета.
При~этом рассматриваются только реальные имена пакетов и их~версий без~учёта имеющихся записей \EN{provides}.
}

\item {
Пакет с~записью \EN{obsoletes} должен содержать также запись \EN{provides},
в~которой имя пакета совпадает с~именем пакета в~записи \EN{obsoletes}.
Если запись \EN{provides} отсутствует, запись \EN{obsoletes} пропускается\footnote 
{
Предполагается, что под действие записи \EN{obsoletes} попадают только реальные имена пакетов, но не записи \EN{provides}. 
В~противном случае, требования наличия одновременно записей \EN{obsoletes} и \EN{provides} будет приводить к~ситуации, в~которой пакеты обновляют сами себя.
}.
}

\item {
Если запись \EN{requires}, по~которой устанавливается пакет, содержит ограничение версии,
то запись \EN{provides} также должна содержать указание версии.
Если указания версии в~записи \EN{provides} нет или версия не~подходит под~требование \EN{requires} замена пакета не~допускается.
}

\item {
Если обнаружено несколько подходящих пакетов для~замены, то выбирается только один из~них на~основе записи \EN{provides},
как это описано ниже. 
}

\end{enumerate}

Если требуемый пакет уже установлен  (включая возможность наличия в~системе пакета с~подходящей записью \EN{provides}), и его версия соответствует требованию версии \EN{requires},
то никакие действия с~ним не~совершаются.
Если пакет не~установлен или его~версия является неподходящей, и не~производилась замена пакета на~основе записи \EN{obsoletes},
то выполняются следующие действия:

\begin{enumerate}

\item {
Если пакет уже установлен, его версия является неподходящей,
но подходящая версия доступна в~репозиториях, то производится его обновление.
}

\item {
Если пакет уже установлен, его версия является неподходящей, и все доступные версии также не~подходят,
то рассматривается множество пакетов на~основе \EN{provides} с~последующей установкой нового варианта.
Причём уже установленный пакет из~системы не~удаляется.
}

\item{
Если пакет не~установлен, но существует его~кандидат, имя которого соответствует заданному, ему отдаётся приоритет, и
 устанавливается  его~подходящая самая свежая версия.
Существующие пакеты, содержащие одноимённые \EN{provides}, не~рассматриваются.
}

\item {
Если запись \EN{requires} имеет ограничение версии, то рассматриваются только пакеты,
соответствующие записи \EN{provides} в~которых содержат информацию о~версии.
Если таких пакетов несколько, то в~указанном порядке последовательно выполняется обработка предустановленного списка приоритетов \EN{provides} (см.~гл.~\ref{advfeatures}),
сортировка по~версии записи \EN{provides} и сортировка по основному имени до~тех~пор, пока один из методов не~исключит неоднозначность выбора.
}

\item {
Если запись \EN{requires} не~имеет ограничений версии, но все подходящие записи \EN{provides} имеют информацию о~версии,
выбор осуществляется таким~же способом, как описано в~предыдущем пункте.
}

\item {
Если запись \EN{requires} не~имеет ограничений версии, и не~все подходящие записи \EN{provides} имеют информацию о~версии,
то выбор производится на~основе предустановленного списка предпочтений или, 
если запись в~списке предпочтений отсутствует, путём выбора последнего элемента в~списке имён пакетов после сортировки по~возрастанию.
}

\end{enumerate}
