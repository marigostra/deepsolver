
\section{Спецификации}

\subsection{Структура пакета}
\label{pkgstruct}

\ds обрабатывает структуру пакета так, как это описано ниже.
Информация о~версии пакета подразумевает наличие следующих компонентов:

\begin{itemize}
\item {Эпоха;}
\item{Версия;}
\item {Релиз.}
\end{itemize}

Эпоха представляет из~себя целое неотрицательное число. 
Версия и релиз могут быть произвольными наборами символов, за~исключением использования символа ``-''.
При~ссылке на~некоторую версию другого пакета указание эпохи и релиза необязательно,
что подразумевает их~произвольное значение 
(отсутствия указания эпохи подразумевает пакет требуемой версии любой эпохи).

Пакет может предоставлять функциональность других пакетов,
указывая информацию об~этом в~тэге \provides.
Имя \provides допускает произвольное значение, не~обязательно совпадающее с~именем какого-либо существующего пакета,
и является, скорее, соглашением, что пакет обладает некоторой совместимостью.
Для~\provides  допускается указание подмножества версии.
Неявными \provides считаются имена всех~файлов, хранимых в~пакете.

Следующие типы отношений допускаются на~множестве пакетов:

\begin{description}

\item[\Requires:]
пакет требует обязательное наличие другого пакета, указанного по~его имени или по~одному из~его \provides.
Допускается указание подмножества версии требуемого пакета.
В~случае указания ограничения версии под~\requires может подходить \provides только дополненный информацией о~версии.

\item[\Conflicts:]
пакет запрещает наличие другого пакета, указанного по~его имени или по~одному из~его \provides.
Допускается указание подмножества версии конфликтуемого пакета.
В~случае указания ограничения версии под~\conflicts может подходить \provides только дополненный информацией о~версии.

\item[\Obsoletes:]
Пакет может указывать, что является обновлением некоторого множества пакетов.
При~установке такого пакета все пакеты, обновлением которых он~является, удаляются из~ОС.
Попытка их~установки после установки обновляющего пакета приводит к~ошибке типа ``установлена более свежая версия''.
Допускается указание подмножества версии обновляемых пакетов.
В~случае указания ограничения версии под~\obsoletes может подходить \provides только дополненный информацией о~версии.

\end{description}

Установка двух пакетов является невозможной, если для~них обнаружены файловые конфликты.
Файловыми конфликтами считаются:

\begin{itemize}

\item {
хранение файлов с~одинаковыми именами, но~с~различной \EN{md5}-суммой или с~различными атрибутами 
(права доступа, идентификаторы владельца и группы, отметка времени создания);
}

\item {
хранение каталогов с~одинаковыми именами, но с~разными атрибутами 
(права доступа, идентификаторы владельца и группы, отметка времени создания).
}

\end{itemize}

Все записи о~конфликтах в~пакете применяются только к~другим пакетам,
т.~е. допускается установка в~систему самоконфликтующих пакетов.
Предполагается, что два~пакета из~разных источников, но с~одинаковыми именами, эпохами, версиями, релизами и отметками времени их~создания являются одним и~тем~же пакетом.

\subsection{Алгоритм выбора пакетов для~установки}

\subsubsection{Порядок выбора пакета по~явному запросу}

При~установке пакета, перечисленного в~команде пользователя, поиск подходящего кандидата ведётся следующим образом:

\begin{enumerate}

\item {
Если пользователь указал имя файла или запросил установку на~основе \EN{URL},
то устанавливается запрошенный пакет без~рассмотрения дополнительных вариантов, описанных ниже.
Если запрошенный пакет уже установлен, но установленная версия новее запрошенной,
то операция не~отменяется.
}

\item {
Если существует пакет, имя которого точно соответствует запрошенному, то выбирается его самая свежая версия, удовлетворяющая ограничениям. 
Прочие варианты, описанные ниже, не~рассматриваются.
%%RFC: Реализация переходит к~обработке provides также в~случае существования пакета, но при~наличии невыполнимых ограничений версий.
}

\item {
Если существуют только пакеты, подходящие под~запрос на~основе записей \EN{provides},
то выбирается только один из~них следующим образом:
}

\begin{itemize}

\item {
если пользователь наложил ограничение версии, то рассматриваются только пакеты,
соответствующие записи \provides в~которых содержат информацию о~версии.
Если таких пакетов несколько, то в~указанном порядке последовательно выполняется обработка предустановленного списка приоритетов \provides, %%FIXME: Ссылка на главу конфигурирования
сортировка по~версии записи \provides и сортировка по основному имени до~тех~пор, пока один из методов не~исключит неоднозначность выбора;
}

\item {
если пользователь ограничения версии не~наложил, но все подходящие записи \provides имеют информацию о~версии,
выбор осуществляется таким~же способом, как описано в~предыдущем пункте;
}

\item {
если пользователь ограничения версии не~наложил, и не~все подходящие записи \provides имеют информацию о~версии,
то выбор производится на~основе предустановленного списка предпочтений или, 
если запись в~списке предпочтений отсутствует, путём выбора последнего элемента в~списке имён пакетов после сортировки по~возрастанию.
}

\end{itemize}
\end{enumerate}

\subsection{Репозитории пакетов}
\label{repo_format}

Репозиторий пакетов \ds может содержать два~уровня разбиения.
Назначение каждого из~них администратор должен определить самостоятельно.
Если в~качестве одного из~них выбрано разделение по~архитектуре процессора, то рекомендуется использовать для~этого первый уровень.

На~первом уровне разделение ведётся по~имени каталога. 
Каждый каталог должен иметь имя, содержащее только латинские буквы, цифры, дефис и знак подчёркивания.
На~втором уровне разбиение производится путём создания нескольких каталогов,
имена которых присвоены по~следующим правилам:

\begin{enumerate}  

\item {
\PATH{base.имя} --- каталог для~индексов компоненты репозитория;
}

\item {
\PATH{RPMS.имя} --- каталог для~бинарных пакетов;
}

\item {
\PATH{SRPMS.имя} --- каталог для~пакетов с~исходными текстами.
}

\end{enumerate}

Например, если требуется создать репозиторий для~архитектур \EN{noarch} и \EN{i586}, 
для~двух компонент \EN{main} и \EN{debuginfo},
то перечень каталогов должен быть следующим:

\begin{itemize}

\item{\PATH{i586/base.main};}
\item{\PATH{i586/base.debuginfo};}
\item{\PATH{i586/RPMS.main};}
\item{\PATH{i586/RPMS.debuginfo};}
\item{\PATH{i586/SRPMS.main};}
\item{\PATH{i586/SRPMS.debuginfo};}
\item{\PATH{noarch/base.main};}
\item{\PATH{noarch/base.debuginfo};}
\item{\PATH{noarch/RPMS.main};}
\item{\PATH{noarch/RPMS.debuginfo};}
\item{\PATH{noarch/SRPMS.main};}
\item{\PATH{noarch/SRPMS.debuginfo}.}

\end{itemize}

Каталог \PATH{base.*}, предназначенный для~хранения индекса, должен содержать следующие файлы:

\begin{itemize}

\item{\PATH{info} --- информационный файл с~параметрами индекса;}
\item{\PATH{.rpms.complete.data} --- вспомогательный файл, не~предназначенный для~загрузки пользователями, с~информацией для~повторной фильтрации \provides;}
\item{\PATH{rpms.data} --- основной список пакетов с~информацией о~зависимостях между ними;}
\item{\PATH{rpms.descr.data} --- список пакетов с~расширенными описаниями;}
\item{\PATH{rpms.filelist.data} --- списки файлов для~каждого бинарного пакета;}
\item{\PATH{srpms.data} --- основная информация о~пакетах с~исходными текстами;}
\item{\PATH{srpms.descr.data} --- список пакетов с~исходными текстами, содержащий расширенную информацию.}

\end{itemize}
